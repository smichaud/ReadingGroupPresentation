\section{FAB-MAP 1.0}

\subsection{Introduction}
\begin{frame}{Let's talk about that article}
   \begin{enumerate}
       \item But do they achieve that ???
       \note[item]{Just for your information, this is what it stands for}
       \note[item]{Already introduced the main idea that it is place recognition + detect new places = SLAM}
   \end{enumerate} 
\end{frame}

%%%%%%%%%%%%%%%%%%%%%%%%%%%%%%%%%%%%%%%%%%%%%%%%%%%%%%%%%%%%%%%%%%%%%%%%%%%%%%%%

\begin{frame}
    \frametitle{FAB-MAP 2008}
    \begin{itemize}
        \item 100\% precision, because false positive break the shit + update appearance model with new sample of known place !
        \item What is perceptual aliasing (with figure)
        \item CLT for visual words co-occurence
    \end{itemize}
\end{frame}

\begin{frame}
    \frametitle{Chow Liu Tree (some slide about it)}
    A further salient aspect of the data is that visual words do not occur independently – indeed, word occurrence tends to be highly correlated. For example, words asso-ciated with car wheels and car doors are likely to be observed simultaneously. We capture these dependencies by learning a tree-structured Bayesian network using the ChowLiu algorithm (Chow and Liu 1968),which yields the optimal approximation to the joint distribution over word occurrence within the space of tree-structured networks.
\end{frame}

\begin{frame}
    \frametitle{Results}
    \begin{itemize}
        \item Talk about processing time in results
    \end{itemize}
\end{frame}
