\section{Conclusion}

\subsection{Usage Note}
\begin{frame}
    \frametitle{Binaries and OpenFABMAP}
    \begin{itemize}
        \item There is FAB-MAP binaries available (2 codebooks)
        \item OpenFABMAP is an implementation of FAB-MAP 1.0
        \item Differences (binary BoW, size of descriptor, probabilistic equations)
        \item Customizable (especially useful for codebook)
        \item Can change keypoints detector and descriptors (STAR worked best for them)
        \item OpenCV, ROS
    \end{itemize}
    \note[item]{Did my own test on Openfabmap...}
\end{frame}

\subsection{FAB-MAP 1.0}
\begin{frame}{Conclusion (FAB-MAP 1.0)}
    \begin{itemize}
        \item Probabilistic framework for place recognition
        \item Appearance-based (visual bag of words)
        \item Capture the co-occurrence of the words (Chow-Liu tree)
        \item Deal with new places (normalization term)
        \item Handle thousands of places in real time
        \item Detect enough loop closure for real applications, without false positives
    \end{itemize}
\end{frame}

\subsection{FAB-MAP 2.0}
\begin{frame}{Conclusion (FAB-MAP 2.0)}
    \begin{itemize}
        \item Improved version 1.0 for \textbf{scalability}
        \item Adapted for \textbf{inverted index}
        \item Added \textbf{geometric verification}
        \item Changed \textbf{k-means} (approximated, initialization, merged)
        \item Can only use \textbf{single observation}
        \item Validated on a \textbf{1000 km dataset}
    \end{itemize}
\end{frame}
