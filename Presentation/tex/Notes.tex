\section{Notes}

\subsection{Notes}

\begin{frame}
    \frametitle{Other publications}
    \begin{itemize}
        \item Mainly the topic of research of P. Newman and Cummins
        \item There is 2008 ICRA and IJRR, 2.0 in 2009 Robotics: Science and Systems and 2011 IJRR
        \item Some extra: 2010 Accelerating..., 2010 fabmap 3d...
    \end{itemize}
\end{frame}

\begin{frame}
    \frametitle{FABMAP 2008}
    \begin{itemize}
        \item Fast Appearance-Based Mapping (FAB-MAP)
        \item What/Why place recognition (SLAM)
        \item Quick overview of SLAM (video of SLAM in action, figure show loop closure)
        \item 100\% precision, because false positive break the shit + update appearance model with new sample of known place !
        \item What is perceptual aliasing (with figure)
        \item CLT for visual words co-occurence
    \end{itemize}
\end{frame}

\begin{frame}
    \frametitle{Chow Liu Tree (some slide about it)}
    A further salient aspect of the data is that visual words do not occur independently – indeed, word occurrence tends to be highly correlated. For example, words asso-ciated with car wheels and car doors are likely to be observed simultaneously. We capture these dependencies by learning a tree-structured Bayesian network using the ChowLiu algorithm (Chow and Liu 1968),which yields the optimal approximation to the joint distribution over word occurrence within the space of tree-structured networks.
\end{frame}

\begin{frame}
    \frametitle{Results}
    \begin{itemize}
        \item Talk about processing time in results
    \end{itemize}
\end{frame}

\begin{frame}
    \frametitle{FABMAP 2.0}
    \begin{itemize}
        \item I just want to give an intuition of the work for 2.0 (interesting to see the challenges to scale)
        \item Pretend to be topological (as oposed to metric)... it contains discrete places and can add new ones == create maps
        \item loop-closure detection, multi-session mapping and kidnapped robot problems. The approach is thus complementary to metric SLAM methods that are typically challenged by these
        \item Inverted index for scalability
    \end{itemize}
\end{frame}

\begin{frame}
    \frametitle{FABMAP 2.0}
    \begin{itemize}
        \item We describe a formulation which pre-serves almost all the key features of our earlier model, but allows for the exploitation of the sparsity of visual word data to achieve large reductions in computation and mem-ory requirements.
        \item validate the work on a 1000km data set; (show the SLAM figure)... Really aimed at large scale (other with 66km but they have the biggest)
    \end{itemize}
\end{frame}

\begin{frame}
    \frametitle{FABMAP 2.0}
    At time k, our map of the environment is a collection of nk discrete and disjoint locations Lk = ?L1, ... ,Lnk ?. Each of these locations has an associated appearance model, whichwe parameterize in terms of unobservable ‘scene elements’, eq. A detector, D, yields visual word observations which are noisy measurements of the existence of the under- lying scene element eq. The appearance model of a location in the map is our belief about the existence of each scene element at that location:
\end{frame}

\begin{frame}
    \frametitle{FABMAP 2.0}
    \begin{itemize}
        \item It deals with negative example to enable inverted index.
        \item Some information lost vs FABMAP 1.0
        \item Cannot add information of observations of a previously seen scene (only the first observation of the place is taken into account)
        \item Pseudocode for the main likelihood calculation required in FAB-MAP is given in Algorithm 1. The complexity of this implementation is O(\#vocab)
    \end{itemize}
\end{frame}

\begin{frame}
    \frametitle{FABMAP 2.0}
    \begin{itemize}
        \item add geometric verification(100 most likely samples) (essential at large scale)... add a figure to show what it is
        \item Adapt other part for scalable : approx k-means
        \item random init yiels poor results :
    \end{itemize}
\end{frame}

\begin{frame}
    \frametitle{FABMAP 2.0}
    \begin{itemize}
        \item To avoid these effects, we choose the initial cluster centers for k-means using a fixed-radius incremental pre-clustering, where the data points are inspected sequentially, and a new cluster center is initialized for every data point that lies further than a fixed threshold from all existing clus-ters. This is similar to the furthest-first initialization tech-nique (Dasgupta and Long 2005), but more computation-ally tractable for large data sets. (show figure 6,7)
    \end{itemize}
\end{frame}

\begin{frame}
    \frametitle{FABMAP 2.0}
    \begin{itemize}
        \item We also modify k-means by adding a cluster merging heuristic. After each k-means iteration, if any two cluster centers are closer than a fixed threshold, one of the two cluster centers is re-initialized to a random location.
        \item This boosted performance of the system
        \item They ajusted the likelihood to work with the inverted index (to expensive to compute with negative example)
    \end{itemize}
\end{frame}

\begin{frame}
    \frametitle{FABMAP 2.0}
    \begin{itemize}
        \item Ground thruth = GPS, correct match less than 40m... A bit relax (89\% were separated by less than 5 m, and 98\% by less than 10m)
        \item Add result Figure 11
        \item Maybe table 3, indicate that CL vs NB for 100\% precision
        \item 14ms filter update + geo check, 423ms for SURF, 60ms Quantization
        \item 4400 times faster than 1.0, spase representation lead to O(1) instead of O(\#vocab)
    \end{itemize}
\end{frame}

\begin{frame}
    \frametitle{OpenFABMAP}
    \begin{itemize}
        \item Quick note: Let them know that there is differences between FABMAP and OpenFABMAP implementation (binary bow, size of descriptors, probabilistic equations...)
        \item There is fabmap binaries available
        \item Codebook and CLT generation
        \item Can change keypoints detector and descriptors (STAR best for them)
        \item Does not have implementations 2.0
    \end{itemize}
\end{frame}

\begin{frame}
    \frametitle{OpenFABMAP}
    \begin{itemize}
        \item Tested on 10 dataset (single cam, omnidirectional)
        \item At best reported 16\% recall for 100\% precision (show figure)
        \item Show little demo with my dataset
    \end{itemize}
\end{frame}
